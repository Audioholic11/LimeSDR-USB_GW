\subsection{i2c\+\_\+opencores.\+c}
\label{software_2lms__ctr__bsp_2drivers_2src_2i2c__opencores_8c_source}\index{/home/erik/git/dev/\+Lime\+S\+D\+R-\/\+U\+S\+B\+\_\+\+G\+W/software/lms\+\_\+ctr\+\_\+bsp/drivers/src/i2c\+\_\+opencores.\+c@{/home/erik/git/dev/\+Lime\+S\+D\+R-\/\+U\+S\+B\+\_\+\+G\+W/software/lms\+\_\+ctr\+\_\+bsp/drivers/src/i2c\+\_\+opencores.\+c}}

\begin{DoxyCode}
00001 
00002 \textcolor{preprocessor}{#include "alt_types.h"}
00003 \textcolor{preprocessor}{#include "i2c\_opencores\_regs.h"}
00004 \textcolor{preprocessor}{#include "i2c\_opencores.h"}
00005 
00006 \textcolor{comment}{// #define I2C\_DEBUG}
00007 \textcolor{comment}{//int I2C\_init(alt\_u32 base,alt\_u32 clk, alt\_u32 speed)}
00008 \textcolor{comment}{//int I2C\_start(alt\_u32 base, alt\_u32 add, alt\_u32 write);}
00009 \textcolor{comment}{//alt\_u32 I2C\_read(alt\_u32 base);}
00010 \textcolor{comment}{//int I2C\_write(alt\_u32 base, alt\_u8 data);}
00011 \textcolor{comment}{//int I2C\_stop(alt\_u32 base);}
00012 
00013 \textcolor{comment}{/* these functions are polled only.  */}
00014 \textcolor{comment}{/* all functions wait until the I2C is done before exiting */}
00015 
00016 
00017 \textcolor{comment}{/****************************************************************}
00018 \textcolor{comment}{int I2C\_init}
00019 \textcolor{comment}{            This function inititlizes the prescalor for the scl}
00020 \textcolor{comment}{            and then enables the core. This must be run before}
00021 \textcolor{comment}{            any other i2c code is executed}
00022 \textcolor{comment}{inputs}
00023 \textcolor{comment}{      base = the base address of the component}
00024 \textcolor{comment}{      clk = freuqency of the clock driving this component  ( in Hz)}
00025 \textcolor{comment}{      speed = SCL speed ie 100K, 400K ...            (in Hz)}
00026 \textcolor{comment}{15-OCT-07 initial release}
00027 \textcolor{comment}{*****************************************************************/}
00028 \textcolor{keywordtype}{void} I2C_init(alt_u32 base,alt_u32 clk,alt_u32 speed)
00029 \{
00030   alt_u32 prescale = (clk/( 5 * speed))-1;
00031 \textcolor{preprocessor}{#ifdef  I2C\_DEBUG}
00032         printf(\textcolor{stringliteral}{" Initializing  I2C at 0x%x, \(\backslash\)n\(\backslash\)twith clock speed 0x%x \(\backslash\)n\(\backslash\)tand SCL speed 0x%x \(\backslash\)n\(\backslash\)tand
       prescale 0x%x\(\backslash\)n"},base,clk,speed,prescale);
00033 \textcolor{preprocessor}{#endif}
00034   IOWR_I2C_OPENCORES_CTR(base, 0x00); \textcolor{comment}{/* turn off the core*/}
00035 
00036   IOWR_I2C_OPENCORES_CR(base, I2C_OPENCORES_CR_IACK_MSK); \textcolor{comment}{/* clearn any pening IRQ*/}
00037 
00038   IOWR_I2C_OPENCORES_PRERLO(base, (0xff & prescale));  \textcolor{comment}{/* load low presacle bit*/}
00039 
00040   IOWR_I2C_OPENCORES_PRERHI(base, (0xff & (prescale>>8)));  \textcolor{comment}{/* load upper prescale bit */}
00041 
00042   IOWR_I2C_OPENCORES_CTR(base, I2C_OPENCORES_CTR_EN_MSK); \textcolor{comment}{/* turn on the core*/}
00043 
00044 \}
00045 
00046 \textcolor{comment}{/****************************************************************}
00047 \textcolor{comment}{int I2C\_start}
00048 \textcolor{comment}{            Sets the start bit and then sends the first byte which}
00049 \textcolor{comment}{            is the address of the device + the write bit.}
00050 \textcolor{comment}{inputs}
00051 \textcolor{comment}{      base = the base address of the component}
00052 \textcolor{comment}{      add = address of I2C device}
00053 \textcolor{comment}{      read =  1== read    0== write}
00054 \textcolor{comment}{return value}
00055 \textcolor{comment}{       0 if address is acknowledged}
00056 \textcolor{comment}{       1 if address was not acknowledged}
00057 \textcolor{comment}{15-OCT-07 initial release}
00058 \textcolor{comment}{*****************************************************************/}
00059 \textcolor{keywordtype}{int} I2C_start(alt_u32 base, alt_u32 add, alt_u32 read)
00060 \{
00061 \textcolor{preprocessor}{#ifdef  I2C\_DEBUG}
00062         printf(\textcolor{stringliteral}{" Start  I2C at 0x%x, \(\backslash\)n\(\backslash\)twith address 0x%x \(\backslash\)n\(\backslash\)tand read 0x%x \(\backslash\)n\(\backslash\)tand prescale 0x%x\(\backslash\)n"},base,
      add,read);
00063 \textcolor{preprocessor}{#endif}
00064 
00065           \textcolor{comment}{/* transmit the address shifted by one and the read/write bit*/}
00066   IOWR_I2C_OPENCORES_TXR(base, ((add<<1) + (0x1 & read)));
00067 
00068           \textcolor{comment}{/* set start and write  bits which will start the transaction*/}
00069   IOWR_I2C_OPENCORES_CR(base, I2C_OPENCORES_CR_STA_MSK | I2C_OPENCORES_CR_WR_MSK );
00070 
00071           \textcolor{comment}{/* wait for the trnasaction to be over.*/}
00072   \textcolor{keywordflow}{while}( IORD_I2C_OPENCORES_SR(base) & I2C_OPENCORES_SR_TIP_MSK);
00073 
00074          \textcolor{comment}{/* now check to see if the address was acknowledged */}
00075    \textcolor{keywordflow}{if}(IORD_I2C_OPENCORES_SR(base) & I2C_OPENCORES_SR_RXNACK_MSK)
00076    \{
00077 \textcolor{preprocessor}{#ifdef  I2C\_DEBUG}
00078         printf(\textcolor{stringliteral}{"\(\backslash\)tNOACK\(\backslash\)n"});
00079 \textcolor{preprocessor}{#endif}
00080         \textcolor{keywordflow}{return} (I2C_NOACK);
00081    \}
00082    \textcolor{keywordflow}{else}
00083    \{
00084 \textcolor{preprocessor}{#ifdef  I2C\_DEBUG}
00085         printf(\textcolor{stringliteral}{"\(\backslash\)tACK\(\backslash\)n"});
00086 \textcolor{preprocessor}{#endif}
00087        \textcolor{keywordflow}{return} (I2C_ACK);
00088    \}
00089 \}
00090 
00091 \textcolor{comment}{/****************************************************************}
00092 \textcolor{comment}{int I2C\_read}
00093 \textcolor{comment}{            assumes that any addressing and start}
00094 \textcolor{comment}{            has already been done.}
00095 \textcolor{comment}{            reads one byte of data from the slave.  on the last read}
00096 \textcolor{comment}{            we don't acknowldge and set the stop bit.}
00097 \textcolor{comment}{inputs}
00098 \textcolor{comment}{      base = the base address of the component}
00099 \textcolor{comment}{      last = on the last read there must not be a ack}
00100 \textcolor{comment}{}
00101 \textcolor{comment}{return value}
00102 \textcolor{comment}{       byte read back.}
00103 \textcolor{comment}{15-OCT-07 initial release}
00104 \textcolor{comment}{*****************************************************************/}
00105 alt_u32 I2C_read(alt_u32 base,alt_u32 last)
00106 \{
00107 \textcolor{preprocessor}{#ifdef  I2C\_DEBUG}
00108         printf(\textcolor{stringliteral}{" Read I2C at 0x%x, \(\backslash\)n\(\backslash\)twith last0x%x\(\backslash\)n"},base,last);
00109 \textcolor{preprocessor}{#endif}
00110   \textcolor{keywordflow}{if}( last)
00111   \{
00112                \textcolor{comment}{/* start a read and no ack and stop bit*/}
00113            IOWR_I2C_OPENCORES_CR(base, I2C_OPENCORES_CR_RD_MSK |
00114                I2C_OPENCORES_CR_NACK_MSK | I2C_OPENCORES_CR_STO_MSK);
00115   \}
00116   \textcolor{keywordflow}{else}
00117   \{
00118           \textcolor{comment}{/* start read*/}
00119           IOWR_I2C_OPENCORES_CR(base, I2C_OPENCORES_CR_RD_MSK );
00120   \}
00121           \textcolor{comment}{/* wait for the trnasaction to be over.*/}
00122   \textcolor{keywordflow}{while}( IORD_I2C_OPENCORES_SR(base) & I2C_OPENCORES_SR_TIP_MSK);
00123 
00124          \textcolor{comment}{/* now read the data */}
00125         \textcolor{keywordflow}{return} (IORD_I2C_OPENCORES_RXR(base));
00126 
00127 \}
00128 
00129 \textcolor{comment}{/****************************************************************}
00130 \textcolor{comment}{int I2C\_write}
00131 \textcolor{comment}{            assumes that any addressing and start}
00132 \textcolor{comment}{            has already been done.}
00133 \textcolor{comment}{            writes one byte of data from the slave.  }
00134 \textcolor{comment}{            If last is set the stop bit set.}
00135 \textcolor{comment}{inputs}
00136 \textcolor{comment}{      base = the base address of the component}
00137 \textcolor{comment}{      data = byte to write}
00138 \textcolor{comment}{      last = on the last read there must not be a ack}
00139 \textcolor{comment}{}
00140 \textcolor{comment}{return value}
00141 \textcolor{comment}{       0 if address is acknowledged}
00142 \textcolor{comment}{       1 if address was not acknowledged}
00143 \textcolor{comment}{15-OCT-07 initial release}
00144 \textcolor{comment}{*****************************************************************/}
00145 alt_u32 I2C_write(alt_u32 base,alt_u8 data, alt_u32 last)
00146 \{
00147 \textcolor{preprocessor}{  #ifdef  I2C\_DEBUG}
00148         printf(\textcolor{stringliteral}{" Read I2C at 0x%x, \(\backslash\)n\(\backslash\)twith data 0x%x,\(\backslash\)n\(\backslash\)twith last0x%x\(\backslash\)n"},base,data,last);
00149 \textcolor{preprocessor}{#endif}
00150                  \textcolor{comment}{/* transmit the data*/}
00151   IOWR_I2C_OPENCORES_TXR(base, data);
00152 
00153   \textcolor{keywordflow}{if}( last)
00154   \{
00155                \textcolor{comment}{/* start a read and no ack and stop bit*/}
00156            IOWR_I2C_OPENCORES_CR(base, I2C_OPENCORES_CR_WR_MSK |
00157                I2C_OPENCORES_CR_STO_MSK);
00158   \}
00159   \textcolor{keywordflow}{else}
00160   \{
00161           \textcolor{comment}{/* start read*/}
00162           IOWR_I2C_OPENCORES_CR(base, I2C_OPENCORES_CR_WR_MSK );
00163   \}
00164            \textcolor{comment}{/* wait for the trnasaction to be over.*/}
00165   \textcolor{keywordflow}{while}( IORD_I2C_OPENCORES_SR(base) & I2C_OPENCORES_SR_TIP_MSK);
00166 
00167          \textcolor{comment}{/* now check to see if the address was acknowledged */}
00168    \textcolor{keywordflow}{if}(IORD_I2C_OPENCORES_SR(base) & I2C_OPENCORES_SR_RXNACK_MSK)
00169    \{
00170 \textcolor{preprocessor}{#ifdef  I2C\_DEBUG}
00171         printf(\textcolor{stringliteral}{"\(\backslash\)tNOACK\(\backslash\)n"});
00172 \textcolor{preprocessor}{#endif}
00173         \textcolor{keywordflow}{return} (I2C_NOACK);
00174    \}
00175    \textcolor{keywordflow}{else}
00176    \{
00177 \textcolor{preprocessor}{#ifdef  I2C\_DEBUG}
00178         printf(\textcolor{stringliteral}{"\(\backslash\)tACK\(\backslash\)n"});
00179 \textcolor{preprocessor}{#endif}
00180        \textcolor{keywordflow}{return} (I2C_ACK);
00181    \}
00182 
00183 \}
\end{DoxyCode}
